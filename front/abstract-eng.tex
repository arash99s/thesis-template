
% -------------------------------------------------------
%  English Abstract
% -------------------------------------------------------

\begin{latin}

\begin{center}
\textbf{Abstract}
\end{center}
\baselineskip=.8\baselineskip

Local Differential Privacy (LDP) is a leading approach for user data protection, guaranteeing privacy without needing to trust the data aggregator. This concept enables data analysis by adding noise to user data before it is sent to the aggregator. This report focuses on the challenges of maintaining LDP for high-dimensional and evolving data. Fundamental challenges in this domain include the correlation between features, increased data sensitivity, and the rapid consumption of the privacy budget, all of which can severely degrade the accuracy and efficiency of statistical analyses.
The secure application of such data is critical in areas like the Internet of Things (IoT), healthcare, and monitoring systems. This is because the data is often used for key decision-making and the development of smart services, making the preservation of user privacy throughout these processes a top priority. This report categorizes, reviews, and compares previous algorithms and works, while also presenting a novel solution to address the challenges posed by high-dimensional and evolving data.
By leveraging the Haar Transform and local hashing, this solution optimally allocates the privacy budget, thereby reducing excess noise. Furthermore, by managing and reducing the data domain, it resolves issues associated with large domains and enhances the accuracy and efficiency of the processes. Ultimately, the findings of this research provide a practical and efficient framework for analyzing high-dimensional and evolving data, which can be employed as a functional tool in various data-driven systems.

\bigskip\noindent\textbf{Keywords}:
Local Differential Privacy, Evolving Data, High-Dimensional Data, Haar Transform, Local Hashing

\end{latin}
