
% -------------------------------------------------------
%  Abstract
% -------------------------------------------------------


\شروع{وسط‌چین}
\مهم{چکیده}
\پایان{وسط‌چین}
\بدون‌تورفتگی

حریم خصوصی تفاضلی محلی یکی از رویکردهای پیشرو در حفاظت از داده‌های کاربران است که بدون اعتماد به کارپذیر، حریم خصوصی را تضمین می‌کند. این مفهوم با افزودن نوفه به داده‌های کاربران قبل از ارسال به سمت کارپذیر، امکان تحلیل داده‌ها را فراهم می‌سازد. این گزارش بر چالش‌های حفظ حریم خصوصی تفاضلی محلی در داده‌های با ابعاد بالا و در حال تغییر تمرکز دارد. از جمله چالش‌های اساسی در این زمینه، همبستگی میان ویژگی‌ها، افزایش حساسیت داده‌ها و مصرف سریع بودجه حریم خصوصی است که می‌تواند دقت و کارایی تحلیل‌های آماری را به شدت کاهش دهد. استفاده ایمن از این داده‌ها در حوزه‌هایی نظیر اینترنت اشیا، داده‌های سلامت و سیستم‌های نظارتی اهمیت بالایی دارد زیرا این داده‌ها معمولاً در تصمیم‌گیری‌های کلیدی و توسعه خدمات هوشمند مورد استفاده قرار می‌گیرند و حفظ حریم خصوصی کاربران در فرآیندها از اولویت بالایی برخوردار است. در این گزارش ضمن دسته‌بندی، مرور و مقایسه الگوریتم‌ها و کارهای پیشین راه‌کار جدیدی برای حل چالش داده‌های با‌ابعاد بالا و در حال تغییر ارائه شده است. این راهکار با بهره‌گیری از دو روش تبدیل هار و درهم‌سازی محلی، بودجه حریم خصوصی را به صورت بهینه تخصیص داده و موجب کاهش نوفه اضافی می‌شود. همچنین با مدیریت و کاهش دامنه داده‌ها، مشکلات مربوط به دامنه‌های بزرگ را حل کرده و دقت و کارایی فرآیندها را افزایش می‌دهد. در نهایت، نتایج این پژوهش چارچوبی عملی و کارآمد برای تحلیل داده‌های با‌ابعاد بالا و در حال تغییر فراهم می‌کند که می‌تواند به عنوان یک ابزار کاربردی در سیستم‌های مبتنی بر داده به کار گرفته شود.

\پرش‌بلند
\بدون‌تورفتگی \مهم{کلیدواژه‌ها}: 
ﺣﺮﻳﻢ ﺧﺼﻮﺻﻲ ﺗﻔﺎﺿﻠﻲ ﻣﺤﻠﻲ، ﺩﺍﺩﻩﻫﺎﻱ ﺩﺭ ﺣﺎﻝ ﺗﻐﻴﻴﺮ، ﺩﺍﺩﻩﻫﺎﻱ ﺑﺎ ﺍﺑﻌﺎﺩ ﺑﺎﻻ، تبدیل هار، درهم‌سازی محلی
\صفحه‌جدید
