
\فصل{مقدمه}

در دنیای امروز که داده‌ها به بخش مهمی از تصمیم‌گیری‌ها و توسعه سیستم‌های اطلاعاتی تبدیل شده‌اند، حفاظت از حریم خصوصی افراد از اهمیت بالایی برخوردار است. حریم خصوصی تفاضلی\پانویس{Differential Privacy} به عنوان یک چارچوب ریاضی، رویکرد نوینی را برای جلوگیری از افشای اطلاعات حساس افراد در تحلیل داده‌ها ارائه می‌دهد.

انتشار داده‌ها بر مبنای حریم خصوصی تفاضلی اخیراً توجه زیادی را به خود جلب کرده و همچنین الگوریتم‌های متنوعی برای بهبود آن ارائه شده‌است. تعریف ریاضی حریم خصوصی تفاضلی بیان می‌کند که نتیجه هر تحلیل آماری روی داده‌ها، چه شما در آن شرکت کنید و چه نه، یکسان باشد. یعنی اطلاعاتی که از داده‌ها استخراج می‌شود، به گونه‌ای طراحی شده‌است که هیچ فردی نتواند از مشارکت یا عدم مشارکت خود، آسیب یا مزیتی دریافت کند. به عبارت دیگر، حریم خصوصی تفاضلی تضمین می‌کند که داده‌های فردی در برابر تحلیل‌های جمعی محافظت می‌شوند و هیچ گونه اطلاعات خاصی درباره افراد در معرض خطر قرار نمی‌گیرد. این ویژگی باعث می‌شود که افراد با اطمینان بیشتری در تحلیل‌های آماری شرکت کنند.

طی چند سال گذشته، راه حل‌هایی ارائه شده‌است تا هر کاربر بتواند ابتدا روی داده‌های خود نوفه\پانویس{Noise} ایجاد کرده و سپس آن‌ها را به سمت کارپذیر ارسال کند. به این راه حل‌ها، الگوریتم‌های حریم خصوصی تفاضلی\پانویس{Local Differential Privacy} محلی گفته می‌شود. در این الگوریتم‌ها، حتی با وجود غیر‌قابل اعتماد بودن کارپذیر، حریم خصوصی کاربران محفوظ باقی می‌ماند. لازم به ذکر است در انتشار داده‌های خصوصی با ابعاد بالا\پانویس{High Dimensional} با چالش‌هایی مانند روابط پیچیده بین ویژگی‌ها، پیچیدگی محاسباتی بالا و پراکندگی داده‌ها روبه‌رو هستیم. راه حل‌های موجود که با تمرکز روی داده با ابعاد پایین ارائه شده‌اند، بودجه حریم خصوصی را بین همه ویژگی‌ها تقسیم می‌کنند. با پیاده‌سازی این راه حل‌ها روی داده با ابعاد بالا، نوفه در مقیاس بالا تولید شده و سیستم کارایی خود را از دست می‌دهد. از طرفی یکی دیگر از چالش‌های حفظ حریم خصوصی تفاضلی، کار روی داده‌های در حال تغییر\پانویس{Evolving Data} است. یک نمونه بارز در چنین مسائلی، نظارت برخط\پانویس{Online} روی برنامه‌های نرم‌افزاری و گزارش عملکرد آن‌ها است، زیرا داده‌های ارسالی همواره در حال تغییر هستند. پروتکل‌های فعلی جمع‌آوری داده‌ها می‌توانند حریم خصوصی تفاضلی را در داده‌های با دامنه تغییرات محدود ارضا کنند. در نتیجه برای دامنه‌های بزرگ، مانند دامنه تغییرات داده‌ها در اینترنت‌اشیاء\پانویس{Internet of Things}، ناکارآمد خواهند بود. هدف از این پژوهش ارائه راهکاری به منظور حفظ حریم خصوصی کاربران در هنگام انتشار داده‌ها با ابعاد بالا و در حال تغییر است. به منظور ارزیابی عملکرد و کارایی این الگوریتم، ما به تحلیل و محاسبه فراوانی داده‌ها پرداخته و مقدار خطای بدست آمده را با خروجی سایر الگوریتم‌های موجود مقایسه می‌کنیم. این مقایسه، به ما کمک می‌کند تا نقاط قوت و ضعف روش پیشنهادی را شناسایی کرده و در جهت بهینه‌سازی بیشتر آن گام برداریم. در نهایت، نتایج حاصل از این پژوهش می‌تواند به عنوان یک چارچوب کاربردی برای حفظ حریم خصوصی داده‌های کاربران مورد استفاده قرار گیرد.


\قسمت{تعریف مسئله}

اکثر مقالات و راهکارهای پیشین، فقط یکی از دو چالش اصلی که پیش‌تر ذکر شد را مورد بررسی قرار داده‌اند و برای ارزیابی راهکار خود مجموعه داده مختص با یک چالش را گردآوری کرده اند. مقالاتی که روی داده‌های با ابعاد بالا کار کرده‌اند، معمولا اختلاف میانگین یا اختلاف احتمال توزیع\پانویس{Probability Distribution} داده‌ها را به عنوان خطا ارائه می‌دهند. همچنین مقالاتی که روی داده‌های در حال تغییر کار می‌کنند، شمارش داده‌های یکسان را به عنوام معیار در نظر گرفته و سعی می‌کنند خطای مربوط به این معیار را کاهش دهند.

از آنجایی که ما هر دو چالش را مورد بررسی قرار داده‌ایم، مجموعه داده‌ای شامل هر دو نوع داده گردآوری شده‌است. یعنی یک مجموعه داده با ابعاد بالا داریم که بعضی از بعدهای آن دامنه‌ی بزرگی دارند و مدام در حال تغییر هستند. به صورت خلاصه راه‌حل ارائه شده، یک نوفه‌ی خاص را روی تمام ابعاد اعمال می‌کند. سپس به منظور ارزیابی راهکار، خطای دو معیار محاسبه می‌شود. معیار شمارش داده‌ها روی بعدهای در حال تغییر در نظر گرفته شده و همچنین معیار احتمال توزیع داده‌ها روی دیگر ابعاد بررسی می‌شود. 

\قسمت{اهمیت موضوع}

در دنیای امروزی برنامه‌های کاربردی بیشماری وجود دارد که مردم با کمک این برنامه‌ها، زندگی روزمره‌ی خود را سپری می‌کنند. داده‌های ورودی برای این برنامه‌ها شامل داده‌های با ابعاد بالا و در حال تغییر می‌شوند. چندین نوع داده کاربردی مهم وجود دارند که حفظ حریم خصوصی افراد در آن حائز اهمیت است، از جمله:

\شروع{فقرات}
\فقره
سوابق پزشکی، داده‌هایی با ابعاد بالا هستند و معمولاً با جمع‌آوری اطلاعات جدید، در طول زمان تغییر می‌کنند. برای مثال، سابقه پزشکی و داده‌های ژنتیکی یک بیمار ممکن است با هر مراجعه به‌روزرسانی شود.
\فقره
مؤسسات مالی با داده‌هایی مانند سوابق تراکنش‌ها و عوامل بازار مثل قیمت سهام و روندهای بازار سروکار دارند. این مجموعه‌داده‌ به‌صورت پیوسته با انجام تراکنش‌ها، نوسان قیمت سهام و ظهور محصولات مالی جدید، تغییر می‌کنند.
\فقره
سکو‌های رسانه‌ اجتماعی داده‌های ابعاد بالا مانند اطلاعات حساب کاربران، تعاملات و ترجیحات را جمع‌آوری می‌کنند که با فعالیت کاربران در محتوا و سکو، تغییر می‌کنند. 
\فقره
شبکه‌های هوشمند و دستگاه‌های اینترنت اشیا، داده‌هایی از حسگرها و دستگاه‌ها مانند الگوهای مصرف انرژی، عوامل محیطی و وضعیت دستگاه‌ها را جمع‌آوری می‌کنند. لازم به ذکر است که این داده‌ها هم بعد‌های زیادی داردن و هم شامل داده‌های در حال تغییر می‌باشند.
\پایان{فقرات}

تا کنون چندین پیاده‌سازی از الگوریتم‌های حریم خصوصی تفاضلی انجام شده است. به عنوان مثال، گوگل\پانویس{Google} در مرورگر کروم\پانویس{Chrome} از الگوریتم رپور\پانویس{Rappor} بهره می‌گیرد تا شاخص‌های کاربری و تنظیمات حساسی مانند صفحهٔ خانگی را از میلیون‌ها کاربر جمع‌آوری کند؛ بی‌آنکه هویت فردی آنان افشا شود. کروم روزانه حدود ۱۴ میلیون گزارش از کاربران داوطلب دریافت می‌کند و با اتکا به این داده‌ها می‌تواند اطلاعات مفیدی را بدون فاش شدن شناسه‌های شخصی در دسترس تحلیلگران قرار دهد \مرجع{ChromeRapporCitekeyMisc}.

از طرفی مایکروسافت\پانویس{Microsoft} برای جمع‌آوری داده‌های دورسنجی\پانویس{Telemetry Data} از سامانه‌هایش به‌جای ارسالِ داده خام، سازوکاری مبتنی بر حریم خصوصی تفاضلیِ محلی پیاده‌سازی کرده است. این فناوری از سال ۲۰۱۷ روی میلیون‌ها دستگاه فعال شده و اکنون معیارهایی\پانویس{Metric} مثل مدت استفاده از هر برنامه را برای هر بازه‌ی شش‌ساعته ثبت می‌کند \مرجع{MicrosoftTelemetryCitekeyMisc}.  بدین‌ترتیب مایکروسافت تنها به نتایج تحلیل‌های آماری دست می‌یابد و هویت یا الگوی اطلاعات کاربران فاش نمی‌شود.

\قسمت{ادبیات موضوع}

\زیرقسمت{حریم خصوصی}

حریم خصوصی مفهومی است که به حق هر فرد برای کنترل دسترسی دیگران به اطلاعات، ارتباطات و قلمرو شخصی‌اش اشاره می‌کند. یعنی هرکس بتواند خود تصمیم بگیرد چه داده‌هایی، در چه زمان و برای چه کسانی آشکار شود و چه بخش‌هایی از زندگی‌اش از نگاه دیگران دور بماند. این حق نه‌تنها شامل اطلاعات آشکار مانند نشانی، شماره تماس یا سوابق پزشکی است، بلکه ترجیحات، گفت‌وگوهای خصوصی و حتی الگوهای رفتاری ضمنی را نیز در بر می‌گیرد. حریم خصوصی با فراهم کردن فضایی امن، مشارکت آگاهانه و بدون ترس در جامعه دیجیتال را ممکن می‌سازد.

\زیرقسمت{داده‌های با ابعاد بالا}

مقصود از داده‌های با ابعاد بالا مجموعه‌ای از داده‌هاست که هر مشاهده آن شامل شمار زیادی ویژگی یا بُعد است. به بیان دیگر، به‌جای سطرهایی با چند ستون محدود، با رکوردهایی روبه‌رو هستیم که ده‌ها یا صدها ستون دارند و هر ستون جنبه‌ای مجزا از پدیده را وصف می‌کند. این تکثر ابعاد، گرچه امکان استخراج الگوها و اطلاعات نهفته‌ی فراوان را فراهم می‌کند، اما هم‌زمان چالش‌هایی در حفظ حریم خصوصی به همراه می‌آورد. در این پژوهش راهکاری مربوط به حل اینگونه چالش‌ها بررسی و پیاده‌سازی شده‌است.

\زیرقسمت{داده‌های طولی}

داده‌های طولی\پانویس{Longitudinal Data} به اطلاعاتی اطلاق می‌شود که در طول زمان و در بازه‌های متوالی از افراد یا موجودیت‌های یکسان جمع‌آوری می‌گردد. در این نوع داده‌ها، هر فرد در چندین نوبت یک نمونه از اطلاعات خود را برای کارپذیر ارسال می‌کند. به عبارت ساده‌تر، این داده‌ها تکامل و تغییرات یک پدیده را در طول زمان دنبال می‌کنند.

\زیرقسمت{داده‌های در حال تغییر}

داده‌های در حال تغییر، مجموعه‌ای از اطلاعات را شامل می‌شود که مقدار ستون‌های آن با گذر زمان تغییر می‌کنند. برای نمونه، تعداد ثانیه‌های استفاده از یک برنامه کاربردی یا هر شمارنده و مؤلفه‌ی دیگری که امروز ارزشی دارد و با گذشت زمان عوض می‌شود. این نوسانِ مداوم باعث می‌شود سامانه داده را به‌طور پی‌درپی جمع‌آوری و به‌روزرسانی کند. معمولا دامنه تغییر این داده‌ها بالا است و چالش‌هایی در حفظ حریم خصوصی بوجود می‌آید که در ادامه به آن اشاره خواهیم کرد. 

\زیرقسمت{پدیده نفرین ابعاد بالا}
نفرین ابعاد بالا به مجموعه چالش‌ها و مشکلاتی گفته می‌شود که با زیاد شدن تعداد ویژگی‌ها در داده رخ می‌دهد. در واقع هر قدر بُعد‌ها بیشتر می‌شوند، احتمال ارتباط میان ابعاد بالا رفته و سامانه برای حفط حریم خصوصی داده‌‌ها به ناچار باید نوفه‌ی بیشتری اضافه کند. هر چقدر نوفه بیشتر اعمال شود، دقت نتایج تحلیل داده‌ها کاهش میابد. بنابراین در این پژوهش راه‌حلی انتخاب و توسعه یافته است که موازنه\پانویس{Trade off} بین حریم خصوصی و سودمندی\پانویس{Utility} رعایت شود.


\قسمت{اهداف پژوهش}

راه‌حل‌های بیشماری تاکنون ارائه شده‌ است ولی اکثر آنها یا حریم خصوصی را کامل ارضا نمی‌کنند یا سربار بالای زمانی و نوفه‌ای به سامانه اعمال می‌کنند. در نتیجه هزینه‌ی استفاده از الگوریتم برای حفظ حریم خصوصی بالا رفته و سازمان‌ها رغبتی به استفاده از آن نمی‌کنند. بنابراین هدف این پژوهش طراحی و پیاده‌سازی ابزاری سبک، کاربرپسند و سریع است که بتواند با تزریق بهینه نوفه به داده‌های خام، هم‌زمان هم سودمندی را داشته باشیم و هم حریم خصوصی حفظ شود.

\قسمت{ساختار پایان‌نامه}

این پایان‌نامه در شش فصل به شرح زیر ارائه می‌شود.
مفاهیم اولیه و مورد استفاده برای حفظ حریم خصوصی تفاضلی در فصل دوم به اختصار اشاره شده است. 
فصل سوم به مطالعه و بررسی کارهای پیشین مرتبط حریم خصوصی تفاضلی محلی می‌پردازد.
در فصل چهارم، راهکار پیشنهادی توضیح داده شده و جزئیات پیاده‌سازی آن بیان می‌شود.
در فصل پنجم، نحوه ارزیابی راهکار پیشنهادی و مقایسه‌ی آن با دیگر راهکارها شرح داده می‌شود. همچنین نتایج جدیدی که در این پایان‌نامه به‌دست آمده است، ارائه خواهد شد.
فصل ششم به جمع‌بندی کارهای انجام شده در این پژوهش و ارائه‌ی پیشنهادهایی برای انجام کارهای آتی خواهد پرداخت.